\section{Question 3}

\textbf{\textit{Assuming a raised male voice in the conference room, calculate the sound pressure level in the adjacent single office due to sound transmission through both the ductwork and the common partition.}}


The resolution of the SPL in the single office due to the sound transmission of a raised male voice through the ductwork and partition, L\textsubscript{p\textsubscript{male\textsubscript{2}}}, is detailed at 63~Hz in Table~\ref{tbl:SPL_office_example} and further explained below.


\subsection{Step 1: Male voice SPL in conference room}

The first step is to calculate the SPL of the raised male voice in the conference room, L\textsubscript{p\textsubscript{male\textsubscript{1}}}.
This is done by using Equation~\ref{eq:SPL}.
V\textsubscript{1} is found in Table~\ref{tbl:room_dims} and the other variables are detailed in Table~\ref{tbl:SPL_office_example}.


\subsection{Step 2: Sound transmission through a single ductwork}

The second step is to calculate the SPL in the single office due to the sound transmission of the male voice through a single ductwork, L\textsubscript{p\textsubscript{male\textsubscript{2}duct}}.
This has been done in the following manner:
\begin{itemize}
	\item The sound power level entering the duct is calculated using Equation~\ref{eq:power_in}, where L\textsubscript{p} is the SPL in the source room and S is the cross-sectional area of the duct.
		\begin{equation}\label{eq:power_in}
			L_{W in} = L_{p} + 10 log \frac{S}{4}
		\end{equation}
	\item From the conference room to the single office, the man's voice is attenuated six times, as listed below. The calculations of these attenuations are detailed in Table~\ref{tbl:SPL_office_example}.
		\begin{enumerate}
			\item Along the 0.5~m vertical duct
			\item At the branch
			\item Along the 4~m horizontal duct
			\item At the branch
			\item Along the 0.5~m vertical duct
			\item At the duct termination due to end reflection losses
		\end{enumerate}
	\item The sound power level leaving a single ductwork is calculated using Equation~\ref{eq:power_out}.
	\item The SPL in the single office due to the sound transmission of the male voice through a single ductwork, L\textsubscript{p\textsubscript{male\textsubscript{2}duct}}, is calculated using Equation~\ref{eq:SPL}. For this, V\textsubscript{2} is found in Table~\ref{tbl:room_dims}.
\end{itemize}


\subsection{Step 3: Sound transmission through the partition}

The third step is to calculate the SPL in the single office due to the sound transmission of the male voice through the partition, L\textsubscript{p\textsubscript{male\textsubscript{2}partition}}.
For this we re-arrange the Sound Reduction Index (SRI) formula (Equation~\ref{eq:SRI}) to find L\textsubscript{p\textsubscript{male\textsubscript{2}partition}} (Equation~\ref{eq:SRI_rearranged}).
Note that S in Equations~\ref{eq:SRI} and \ref{eq:SRI_rearranged} is the surface area of the common wall in square metres.

	\begin{equation}\label{eq:SRI}
		R = L_1 - L_2 + 10 log \frac{S}{A_2}
	\end{equation}
	
	
	\begin{equation}\label{eq:SRI_rearranged}
		L_2 = L_1 - R + 10 log \frac{S}{A_2}
	\end{equation}

\begin{itemize}
	\item The SRI values of the partition, R, are given in Table~2 of the assignment description.
	\item The surface area of the partition wall, S\textsubscript{partition}, is found in Table~\ref{tbl:room_dims}.
	\item The absorption of the single office, A\textsubscript{2}, is found in Table~\ref{tbl:reverb_office}.
\end{itemize}

The calculation of L\textsubscript{p\textsubscript{male\textsubscript{2}partition}} is detailed in Table~\ref{tbl:SPL_office_example}.


\subsection{Step 4: Sound transmission through partition and both ducts}

The final step is to sum the SPLs in the office due to sound transmission through the partition and both the supply and return ducts.
For this we use Equation~\ref{eq:sum} to calculate the sum of any number of levels.
See detailed calculation of L\textsubscript{p\textsubscript{male\textsubscript{2}}} in Table~\ref{tbl:SPL_office_example}.

	\begin{equation}\label{eq:sum}
		L_T = 10 log \left(10^{L_1/10} + 10^{L_2/10} + 10^{L_3/10} + \ldots\right) = 10 log \left(\sum_{i}10^{L_i/10}\right)
	\end{equation}


\subsection{L\textsubscript{p\textsubscript{male\textsubscript{2}}} results at all frequencies}

The L\textsubscript{p\textsubscript{male\textsubscript{2}}} results are presented at all frequencies in Table~\ref{tbl:SPL_office}.



% Please add the following required packages to your document preamble:
% \usepackage{booktabs}
\begin{table}[htbp]
	\caption{Calculation of the SPL in the single office due to sound transmission of a raised male voice through the ductwork and partition, L\textsubscript{p\textsubscript{male\textsubscript{2}}}.}
	\label{tbl:SPL_office}
	\centering
	\begin{tabular}{@{}l@{\hspace*{0.5em}}lrrrrrrr@{}}% <-- @{} adjusted inter column space
		\toprule
		Step & Frequency (Hz) & 63 & 125 & 250 & 500 & 1000 & 2000 & 4000 \\ \midrule
		\multirow{3}{*}{1 \vast\{ } & L\textsubscript{W\textsubscript{male}} (dB re 10\textsuperscript{-12} W) & 62.29 & 67.43 & 70.00 & 73.53 & 74.71 & 70.00 & 60.00 \\
		& T\textsubscript{1} (s) & 0.48 & 0.44 & 0.47 & 0.49 & 0.37 & 0.32 & 0.30 \\
		& L\textsubscript{p\textsubscript{male\textsubscript{1}}} (dB) & 54.11 & 58.78 & 61.70 & 65.41 & 65.34 & 59.99 & 49.79 \\
		\multirow{11}{*}{2 \Vast\{ } & L\textsubscript{W\textsubscript{in}} (dB re 10\textsuperscript{-12} W) & 38.88 & 43.56 & 46.47 & 50.18 & 50.11 & 44.76 & 34.56 \\
		& Attn\textsubscript{vertical} (dB) & 1.00 & 1.32 & 1.00 & 0.66 & 0.23 & 0.23 & 0.23 \\
		& Attn\textsubscript{vertical $\rightarrow$ horizontal} (dB) & 3.01 & 3.01 & 3.01 & 3.01 & 3.01 & 3.01 & 3.01 \\
		& Attn\textsubscript{horizontal} (dB) & 10.56 & 10.56 & 6.64 & 3.92 & 1.56 & 1.56 & 1.56 \\
		& Attn\textsubscript{horizontal $\rightarrow$ vertical} (dB) & 4.77 & 4.77 & 4.77 & 4.77 & 4.77 & 4.77 & 4.77 \\
		& Attn\textsubscript{vertical} (dB) & 1.00 & 1.32 & 1.00 & 0.66 & 0.23 & 0.23 & 0.23 \\
		& Attn\textsubscript{end reflections} (dB) & 11 & 7 & 3 & 1 & 0 & 0 & 0 \\
		& $\Sigma$Attn (dB) & 31.34 & 27.98 & 19.42 & 14.02 & 9.80 & 9.80 & 9.80 \\
		& L\textsubscript{W\textsubscript{out}} (dB re 10\textsuperscript{-12} W) & 7.53 & 15.57 & 27.05 & 36.16 & 40.31 & 34.96 & 24.76 \\
		& T\textsubscript{2} (s) & 0.47 & 0.41 & 0.46 & 0.48 & 0.36 & 0.31 & 0.30 \\
		& L\textsubscript{p\textsubscript{male\textsubscript{2}duct}} (dB) & 1.26 & 8.70 & 20.65 & 29.96 & 32.91 & 26.92 & 16.52 \\
		\multirow{3}{*}{3 \vast\{ } & R (dB) & 20 & 32 & 45 & 52 & 58 & 64 & 61 \\
		& A\textsubscript{2} (s) & 17.15 & 19.70 & 17.65 & 16.85 & 22.25 & 25.80 & 27.00 \\
		& L\textsubscript{p\textsubscript{male\textsubscript{2}partition}} (dB) & 31.76 & 23.84 & 14.23 & 11.14 & 3.87 & -8.12 & -15.52 \\
		4 & L\textsubscript{p\textsubscript{male\textsubscript{2}}} (dB) & 31.8 & 24.1 & 24.1 & 33.0 & 35.9 & 29.9 & 19.5 \\ \bottomrule
	\end{tabular}
\end{table}

% Please add the following required packages to your document preamble:
% \usepackage{booktabs}
\begin{sidewaystable}[htbp]
	\caption{Details of the calculation of the SPL in the single office due to sound transmission of a raised male voice through the ductwork and partition, L\textsubscript{p\textsubscript{male\textsubscript{2}}}, at 63~Hz.}
	\label{tbl:SPL_office_example}
	\centering
	\begin{tabular}{@{}l@{\hspace*{0.5em}}lrm{10.5cm}@{}}% <-- @{} adjusted inter column space
		\toprule
		Step & Frequency (Hz) & 63 & Notes \\ \midrule
		\multirow{5}{*}{1 \vastt\{ } & L\textsubscript{W\textsubscript{male}} (dB re 10\textsuperscript{-12} W) & 62.29 & Read off the chart in Table~7 of assignment description \\
		& & \multicolumn{1}{l}{} &  \\
		& T\textsubscript{1} (s) & 0.48 & From Table~\ref{tbl:reverb_conf} \\
		& & \multicolumn{1}{l}{} &  \\
		& L\textsubscript{p\textsubscript{male\textsubscript{1}}} (dB) & $62.29 - 10 log \left(\frac{80}{0.48}\right) + 14 = 54.11$ & \begin{tabular}[c]{@{}l@{}}SPL in conference room due to man's voice:\\ $L_{p_{male_1}} = L_{W male} – 10 log \frac{V_1}{T_1} + 14$\end{tabular} \\
		& & \multicolumn{1}{l}{} &  \\
		\multirow{21}{*}{2 \Vastt\{ } & L\textsubscript{W\textsubscript{in}} (dB re 10\textsuperscript{-12} W) & $54.11 + 10 log \left(\frac{0.12}{4}\right) = 38.88$ & \begin{tabular}[c]{@{}l@{}}Sound power level entering single ductwork:\\ $L_{W in} = L_{p_{male_1}} + 10 log \frac{S_{vertical}}{4}$\end{tabular} \\
		& & \multicolumn{1}{l}{} &  \\
		& Attn\textsubscript{vertical} (dB) & $0.5 \times (1 + 1) = 1.00$ & Same as notes for Attn\textsubscript{vertical} in Table~\ref{tbl:BN_conf_example} \\
		& & \multicolumn{1}{l}{} &  \\
		& Attn\textsubscript{vertical $\rightarrow$ horizontal} (dB) & $\left|10 log \frac{0.24}{0.24 + 0.24}\right| = 3.01$ & $\left|10 log \frac{S_{horizontal}}{S_{horizontal} + S_{horizontal}}\right|$ \\
		& & \multicolumn{1}{l}{} &  \\
		& Attn\textsubscript{horizontal} (dB) & $4 \times (1 + 1.64) = 10.56$ & Same as notes for Attn\textsubscript{horizontal} in Table~\ref{tbl:BN_conf_example} but for a duct length of 4~m \\
		& & \multicolumn{1}{l}{} &  \\
		& Attn\textsubscript{horizontal $\rightarrow$ vertical} (dB) & $\left|10 log \frac{0.12}{0.12 + 0.24}\right| = 4.77$ & $\left|10 log \frac{S_{vertical}}{S_{vertical} + S_{horizontal}}\right|$ \\
		& & \multicolumn{1}{l}{} &  \\
		& Attn\textsubscript{vertical} (dB) & $0.5 \times (1 + 1) = 1.00$ & Same as notes for Attn\textsubscript{vertical} in Table~\ref{tbl:BN_conf_example} \\
		& & \multicolumn{1}{l}{} &  \\
		& Attn\textsubscript{end reflections} (dB) & 11 & Given in Table~6 of assignment description for a duct cross-sectional area of 0.3~m $\times$ 0.4~m = 0.12~m\textsuperscript{2} \\
		& & \multicolumn{1}{l}{} &  \\
		& $\Sigma$Attn (dB) & 31.34 & Sum of all attenuations \\
		& & \multicolumn{1}{l}{} &  \\
		& L\textsubscript{W\textsubscript{out}} (dB re 10\textsuperscript{-12} W) & $38.88 - 31.34 = 7.53$ & \begin{tabular}[c]{@{}l@{}}Sound power level leaving single ductwork: $L_{W out} = L_{W in} - \Sigma Attn$\end{tabular} \\
		& & \multicolumn{1}{l}{} &  \\
		& T\textsubscript{2} (s) & 0.47 & From Table~\ref{tbl:reverb_office} \\
		& & \multicolumn{1}{l}{} &  \\
		& L\textsubscript{p\textsubscript{male\textsubscript{2}duct}} (dB) & $7.53 - 10 log \left(\frac{50}{0.47}\right) + 14 = 1.26$ & \begin{tabular}[c]{@{}l@{}}SPL in office due to man's voice: $L_{p_{male_2duct}} = L_{W out} – 10 log \frac{V_2}{T_2} + 14$\end{tabular} \\
		& & \multicolumn{1}{l}{} &  \\
		\multirow{5}{*}{3 \vastt\{ } & R (dB) & 20 & Given in Table~2 of assignment description \\
		& & \multicolumn{1}{l}{} &  \\
		& A\textsubscript{2} (s) & 17.15 & From Table~\ref{tbl:reverb_office} \\
		& & \multicolumn{1}{l}{} &  \\
		& L\textsubscript{p\textsubscript{male\textsubscript{2}partition}} (dB) & $54.11 - 20 + 10 log \left(\frac{10}{17.15}\right) = 31.76$ & \begin{tabular}[c]{@{}l@{}}Re-arrange SRI formula to get SPL after partition:\\ $L_{p_{male_2partition}} = L_{p_{male_1}} – R + 10 log \frac{S_{partition}}{A_2}$\end{tabular} \\
		& & \multicolumn{1}{l}{} &  \\
		4 & L\textsubscript{p\textsubscript{male\textsubscript{2}}} (dB) & $10 log \left(10^{1.26/10} + 10^{1.26/10} + 10^{31.76/10}\right) = 31.8$ & Sum of three SPLs: through supply duct, return duct and partition \\ \bottomrule
	\end{tabular}
\end{sidewaystable}