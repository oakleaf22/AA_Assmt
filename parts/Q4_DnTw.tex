\section{Question 4}

\textbf{\textit{Calculate the weighted standardised level difference (D\textsubscript{nT,w}) between the conference room and single office for voice transmission through both the ductwork and the common partition, and discuss the results in terms of privacy.}}


The resolution of (D\textsubscript{nT,w}) between the conference room and single office for voice transmission through both the ductwork and the common partition is detailed and explained at 63~Hz in Sections~\ref{sec:4.1} to \ref{sec:4.3}.
\ldots


\subsection{Step 1: Calculate D\textsubscript{nT}} \label{sec:4.1}

The first step is to calculate the standardised level difference, D\textsubscript{nT}, through the ductwork, through the partition, and through both the ductwork and the partition.
This is done using Equation~\ref{eq:DnT}.
Table~\ref{tbl:DnT_example} details the D\textsubscript{nT} calculations at 63~Hz and Table~\ref{tbl:DnT} presents the D\textsubscript{nT} results in octave bands between 63~Hz and 4~kHz.

	\begin{equation}\label{eq:DnT}
		D_{nT} = L_1 - L_2 + 10 log \frac{T_2}{0.5}
	\end{equation}

% Please add the following required packages to your document preamble:
% \usepackage{booktabs}
\begin{table}[htbp]
	\caption{Calculation of the DnT between the conference room and single office for voice transmission through the ductwork and partition.}
	\label{tbl:DnT}
	\centering
	\begin{tabular}{@{}lrrrrrrr@{}}
		\toprule
		Frequency (Hz) & 63 & 125 & 250 & 500 & 1000 & 2000 & 4000 \\ \midrule
		L\textsubscript{p\textsubscript{male\textsubscript{1}}} (dB) & 54.11 & 58.78 & 61.70 & 65.41 & 65.34 & 59.99 & 49.79 \\
		L\textsubscript{p\textsubscript{male\textsubscript{2}duct}} (dB) & 1.26 & 8.70 & 20.65 & 29.96 & 32.91 & 26.92 & 16.52 \\
		L\textsubscript{p\textsubscript{male\textsubscript{2}ducts}} (dB) & 4.27 & 11.71 & 23.66 & 32.97 & 35.92 & 29.93 & 19.53 \\
		L\textsubscript{p\textsubscript{male\textsubscript{2}partition}} (dB) & 31.76 & 23.84 & 14.23 & 11.14 & 3.87 & -8.12 & -15.52 \\
		L\textsubscript{p\textsubscript{male\textsubscript{2}}} (dB) & 31.77 & 24.10 & 24.13 & 33.00 & 35.92 & 29.93 & 19.53 \\
		T\textsubscript{2} (s) & 0.47 & 0.41 & 0.46 & 0.48 & 0.36 & 0.31 & 0.30 \\
		D\textsubscript{nT} through ductwork (dB) & 49.6 & 46.2 & 37.6 & 32.2 & 28.0 & 28.0 & 28.0 \\
		D\textsubscript{nT} through partition (dB) & 22.1 & 34.1 & 47.1 & 54.1 & 60.1 & 66.1 & 63.1 \\
		D\textsubscript{nT} through ductwork and partition (dB) & 22.1 & 33.8 & 37.2 & 32.2 & 28.0 & 28.0 & 28.0 \\ \bottomrule
	\end{tabular}
\end{table}

% Please add the following required packages to your document preamble:
% \usepackage{booktabs}
\begin{sidewaystable}[htbp]
	\caption{Details of the calculation of the DnT between the conference room and single office for voice transmission through the ductwork and partition at 63~Hz.}
	\label{tbl:DnT_example}
	\centering
	\begin{tabular}{@{}lrl@{}}
		\toprule
		Frequency (Hz) & 63 & Notes \\ \midrule
		L\textsubscript{p\textsubscript{male\textsubscript{1}}} (dB) & 54.11 & From Table~\ref{tbl:SPL_office} \\
		L\textsubscript{p\textsubscript{male\textsubscript{2}duct}} (dB) & 1.26 & From Table~\ref{tbl:SPL_office} \\
		L\textsubscript{p\textsubscript{male\textsubscript{2}ducts}} (dB) & 1.26 + 3 = 4.27 & L\textsubscript{p\textsubscript{male\textsubscript{2}ducts}} = L\textsubscript{p\textsubscript{male\textsubscript{2}duct}} + 3 \\
		L\textsubscript{p\textsubscript{male\textsubscript{2}partition}} (dB) & 31.76 & From Table~\ref{tbl:SPL_office} \\
		L\textsubscript{p\textsubscript{male\textsubscript{2}}} (dB) & 31.77 & From Table~\ref{tbl:SPL_office} \\
		T\textsubscript{2} (s) & 0.47 & From Table~\ref{tbl:reverb_office} \\
		D\textsubscript{nT} through ductwork (dB) & $54.11 - 4.27 + 10 log \frac{0.47}{0.5} = 49.6$ & $D_{nT} = L_{p_{male_1}} - L_{p_{male_2}ducts} + 10 log \frac{T_2}{0.5}$ \\
		D\textsubscript{nT} through partition (dB) & $54.11 - 31.76 + 10 log \frac{0.47}{0.5} = 22.1$ & $D_{nT} = L_{p_{male_1}} - L_{p_{male_2}partition} + 10 log \frac{T_2}{0.5}$ \\
		D\textsubscript{nT} through ductwork and partition (dB) & $54.11 - 31.77 + 10 log \frac{0.47}{0.5} = 22.1$ & $D_{nT} = L_{p_{male_1}} - L_{p_{male_2}} + 10 log \frac{T_2}{0.5}$ \\ \bottomrule
	\end{tabular}
\end{sidewaystable}


\subsection{Step 2: Calculate D\textsubscript{nT,w}} \label{sec:4.2}

The second step is to calculate D\textsubscript{nT,w} through the ductwork, through the partition, and through both the ductwork and the partition.
However, before we can do this, we need to extrapolate the third octave band values of D\textsubscript{nT}.
The extrapolation calculation is demonstrated in Table~\ref{tbl:DnT_extrapolation}.

How to get DnTw...

% Please add the following required packages to your document preamble:
% \usepackage{booktabs}
\begin{table}[htbp]
	\caption{A sample calculation of the extrapolation of the third octave band values of D\textsubscript{nT} through the ductwork.}
	\label{tbl:DnT_extrapolation}
	\centering
	\begin{tabular}{@{}rrl@{}}
		\toprule
		Frequency (Hz) & D\textsubscript{nT} through ductwork (dB) & Notes \\ \midrule
		63 & 50 & From Table~\ref{tbl:DnT} \\
		80 & $50 + \frac{46 - 50}{3} = 48$ &  \\
		100 & $50 + \frac{2 \times (46 - 50)}{3} = 47$ &  \\
		125 & 46 & From Table~\ref{tbl:DnT} \\
		160 & $46 + \frac{38 - 46}{3} = 43$ &  \\
		200 & $46 + \frac{2 \times (38 - 46)}{3} = 40$ &  \\
		250 & 38 & From Table~\ref{tbl:DnT} \\
		\ldots & \ldots & \ldots \\ \bottomrule
	\end{tabular}
\end{table}
	
% Please add the following required packages to your document preamble:
% \usepackage{booktabs}
\begin{sidewaystable}[htbp]
	\caption{Calculation of D\textsubscript{nT,w} through the ductwork only.}
	\label{tbl:DnT_ductwork}
	\centering
	\begin{tabu} to \textwidth {@{}X[r]X[r]X[r]X[r]X[r]X[r]X[r]X[r]X[r]X[r]X[r]X[r]X[r]X[r]@{}}
		\toprule
		Frequency (Hz) & D\textsubscript{nT} (dB) & 52 dB Curve & Diff & 42 dB Curve & Diff & 32 dB Curve & Diff & 27 dB Curve & Diff & 28 dB Curve & Diff & \cellcolor{yellow} 29 dB Curve & Diff \\ \midrule
		63 & 50 & - & - & - & - & - & - & - & - & - & - & - & - \\
		80 & 48 & - & - & - & - & - & - & - & - & - & - & - & - \\
		100 & 47 & 33 & - & 23 & - & 13 & - & 8 & - & 9 & - & 10 & - \\
		125 & 46 & 36 & - & 26 & - & 16 & - & 11 & - & 12 & - & 13 & - \\
		160 & 43 & 39 & - & 29 & - & 19 & - & 14 & - & 15 & - & 16 & - \\
		200 & 40 & 42 & 2 & 32 & - & 22 & - & 17 & - & 18 & - & 19 & - \\
		250 & 38 & 45 & 7 & 35 & - & 25 & - & 20 & - & 21 & - & 22 & - \\
		315 & 36 & 48 & 12 & 38 & 2 & 28 & - & 23 & - & 24 & - & 25 & - \\
		400 & 34 & 51 & 17 & 41 & 7 & 31 & - & 26 & - & 27 & - & 28 & - \\
		500 & 32 & 52 & 20 & 42 & 10 & 32 & - & 27 & - & 28 & - & 29 & - \\
		630 & 31 & 53 & 22 & 43 & 12 & 33 & 2 & 28 & - & 29 & - & 30 & - \\
		800 & 29 & 54 & 25 & 44 & 15 & 34 & 5 & 29 & - & 30 & 1 & 31 & 2 \\
		1000 & 28 & 55 & 27 & 45 & 17 & 35 & 7 & 30 & 2 & 31 & 3 & 32 & 4 \\
		1250 & 28 & 56 & 28 & 46 & 18 & 36 & 8 & 31 & 3 & 32 & 4 & 33 & 5 \\
		1600 & 28 & 56 & 28 & 46 & 18 & 36 & 8 & 31 & 3 & 32 & 4 & 33 & 5 \\
		2000 & 28 & 56 & 28 & 46 & 18 & 36 & 8 & 31 & 3 & 32 & 4 & 33 & 5 \\
		2500 & 28 & 56 & 28 & 46 & 18 & 36 & 8 & 31 & 3 & 32 & 4 & 33 & 5 \\
		3150 & 28 & 56 & 28 & 46 & 18 & 36 & 8 & 31 & 3 & 32 & 4 & 33 & 5 \\
		4000 & 28 & - & - & - & - & - & - & - & - & - & - & - & - \\ \midrule
		\multicolumn{3}{r}{Sum of differences (dB)} & 271 &  & 153 &  & 54 &  & 17 &  & 23 &  & \cellcolor{yellow} 30 \\ \bottomrule
	\end{tabu}
\end{sidewaystable}

% Please add the following required packages to your document preamble:
% \usepackage{booktabs}
\begin{table}[htbp]
	\caption{Calculation of D\textsubscript{nT,w} through the partition only.}
	\label{tbl:DnT_partition}
	\centering
	\begin{tabular}{@{}rrrrrrrrrr@{}}
		\toprule
		Frequency (Hz) & D\textsubscript{nT} (dB) & 52 dB Curve & Diff & 62 dB Curve & Diff & 57 dB Curve & Diff & \cellcolor{yellow} 56 dB Curve & Diff \\ \midrule
		63 & 22 & - & - & - & - & - & - & - & - \\
		80 & 26 & - & - & - & - & - & - & - & - \\
		100 & 30 & 33 & 3 & 43 & 13 & 38 & 8 & 37 & 7 \\
		125 & 34 & 36 & 2 & 46 & 12 & 41 & 7 & 40 & 6 \\
		160 & 38 & 39 & 1 & 49 & 11 & 44 & 6 & 43 & 5 \\
		200 & 43 & 42 & - & 52 & 9 & 47 & 4 & 46 & 3 \\
		250 & 47 & 45 & - & 55 & 8 & 50 & 3 & 49 & 2 \\
		315 & 49 & 48 & - & 58 & 9 & 53 & 4 & 52 & 3 \\
		400 & 52 & 51 & - & 61 & 9 & 56 & 4 & 55 & 3 \\
		500 & 54 & 52 & - & 62 & 8 & 57 & 3 & 56 & 2 \\
		630 & 56 & 53 & - & 63 & 7 & 58 & 2 & 57 & 1 \\
		800 & 58 & 54 & - & 64 & 6 & 59 & 1 & 58 & - \\
		1000 & 60 & 55 & - & 65 & 5 & 60 & - & 59 & - \\
		1250 & 62 & 56 & - & 66 & 4 & 61 & - & 60 & - \\
		1600 & 64 & 56 & - & 66 & 2 & 61 & - & 60 & - \\
		2000 & 66 & 56 & - & 66 & - & 61 & - & 60 & - \\
		2500 & 65 & 56 & - & 66 & 1 & 61 & - & 60 & - \\
		3150 & 64 & 56 & - & 66 & 2 & 61 & - & 60 & - \\
		4000 & 63 & - & - & - & - & - & - & - & - \\ \midrule
		\multicolumn{3}{r}{Sum of differences (dB)} & 5 &  & 105 &  & 41 &  & \cellcolor{yellow} 31 \\ \bottomrule
	\end{tabular}
\end{table}

% Please add the following required packages to your document preamble:
% \usepackage{booktabs}
\begin{table}[htbp]
	\caption{Calculation of D\textsubscript{nT,w} through both the ductwork and partition.}
	\label{tbl:DnTw_both}
	\centering
	\begin{tabular}{@{}rrrrrr@{}}
		\toprule
		Frequency (Hz) & D\textsubscript{nT} (dB) & 52 dB Curve & Diff & \cellcolor{pink} 29 dB Curve & Diff \\ \midrule
		63 & 22 & - & - & - & - \\
		80 & 26 & - & - & - & - \\
		100 & 30 & 33 & 3 & 10 & - \\
		125 & 34 & 36 & 2 & 13 & - \\
		160 & 35 & 39 & 4 & 16 & - \\
		200 & 36 & 42 & 6 & 19 & - \\
		250 & 37 & 45 & 8 & 22 & - \\
		315 & 36 & 48 & 12 & 25 & - \\
		400 & 34 & 51 & 17 & 28 & - \\
		500 & 32 & 52 & 20 & 29 & - \\
		630 & 31 & 53 & 22 & 30 & - \\
		800 & 29 & 54 & 25 & 31 & 2 \\
		1000 & 28 & 55 & 27 & 32 & 4 \\
		1250 & 28 & 56 & 28 & 33 & 5 \\
		1600 & 28 & 56 & 28 & 33 & 5 \\
		2000 & 28 & 56 & 28 & 33 & 5 \\
		2500 & 28 & 56 & 28 & 33 & 5 \\
		3150 & 28 & 56 & 28 & 33 & 5 \\
		4000 & 28 & - & - & - & - \\ \midrule
		\multicolumn{3}{r}{Sum of differences (dB)} & 286 &  & \cellcolor{pink} 30 \\ \bottomrule
	\end{tabular}
\end{table}


\subsection{Step 3: Calculate the SPL in the single office due to fan noise transmitted through the ductwork} \label{sec:4.3}



\subsection{Conversation privacy} \label{sec:4.4}

